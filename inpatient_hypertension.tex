\documentclass{tufte-handout}
\title{Inpatient Hypertension Management}
\author{Andrew Zimolzak, MD, MMSc}
\date{April 9, 2018}

\begin{document}

\maketitle

\marginnote{Licensed under Creative Commons BY-NC-SA 4.0. You are free
  to share and adapt this material for noncommercial purposes if you
  credit the author and distribute under the same license. Start
  sharing/adapting at github.com/zimolzak/inpatient-hypertension}

\section{Quotes from the most pertinent paper I found}

``Current evidence-based hypertension guidelines do not specifically
address inpatient hypertension. \ldots{}tendency for inpatient
providers to overreact to asymptomatic elevated BP which is often
treated with intravenous medications offering little benefit and
risking potential harm. [C]alls regarding elevated BP are often a bane
for on-call physicians who may feel compelled to react. Once it is
established that elevated BP is nonemergent, hospital providers should
consider several factors that might contribute to elevated BP.'' Treat
anxiety, pain, etc. Treat volume overload if present. Restart meds if
on hold.

Finally, \emph{if BP consistently elevated > 20 mmHg above age and
  condition appropriate guidelines,} titrate existing meds or add new
ones. Care should be taken, though, to avoid over-titration of BP
medications realizing that long-acting medications take days to weeks
to reach steady-state concentrations. Having described instances of
overtreatment of BP, it is somewhat paradoxical that opportunities for
improvement in care transitions of HTN care are also
apparent. \footnote{Axon \emph{et al.} Curr Cardiol Rep. 2015 Nov;17(11):94.
  PMID: 26362300.}

Guidelines 2017 (oversimplified) say: 135 systolic, 130 if CAD, DM,
CKD, 65 years old, or 10-year risk of 10\%. Therefore if you want one
number, 155.

\section{Other thoughts about inpatient hypertension alone}

Inpatient hypertension is common.\footnote{Axon \emph{et al.}
  Prevalence and management of hypertension in the inpatient setting:
  a systematic review. J Hosp Med. 2011;6:417--22. PMID: 20652961}
Existing (ACC) guidelines don't necessarily say what to do.\footnote{Weder
  AB. Treating acute hypertension in the hospital: a Lacuna in the
  guidelines. Hypertension. 2011; 57:18--20. PMID: 21079044} Recognize
that later stages of heart failure are different (in their BP numbers
and goals), even in euvolemic patients.


%% What risk are we reducing (what time baseline)?
%% How long do various agents take? ACE ARB mnemonics. Beta. Calcium.
%% Others like clonidine. Which classes are good for ``treating a
%% number?''

 

\section{A brief word on hypertensive urgency}

%% Where does the threshold start, where did it used to start? What meds
%% would I use? I don't advocate IV. What would I check for or think
%% about regarding \emph{emergency?} Steep rise should make you more
%% likely to think of emergency. Compare to somebody who is ``used to''
%% that high pressure.

Choice quote: ``Many of these patients have withdrawn from or are
noncompliant with antihypertensive therapy\ldots{}. [T]reated by
reinstitution or intensification of antihypertensive drug therapy and
treatment of anxiety\ldots{}. There is no indication for referral to
the emergency department, immediate reduction in BP in the emergency
department, or hospitalization for such patients.''\footnote{Whelton
  \emph{et al.} 2017 ACC (and 10 other societies) Guideline. Hypertension.
  2017 Nov 13. This is 192 pages with > 200 pp.\ of supplements. Pp.
  137--143 are about emergency/urgency.}

\newpage

\section{Many references, but not as pertinent to gen.\ med.\ wards}

This section is mainly useful if you want to pull the papers cited.

In a registry of 1588 ``patients with acute severe hypertension
requiring hospitalization,'' 64\% required multiple drugs, 60\% had
reelevation to >180, 59\% had end-organ dysfunction during
hospitalization, and 37\% were readmitted within 90 days. Median time
to BP <160 was 4 hours. 4\% had iatrogenic hypotension, 6.9\% died in
hospital, and 11\% died within 90 days.\footnote{Katz \emph{et al.} Am Heart
  J.\ 2009 Oct;158(4):599--606.} Note \emph{ICU or ER setting, and IV
  meds is an inclusion criterion.}

Hospital length of stay for patients for whom hydralazine was ordered
was 12.0 days and 7.1 days for those who did not receive a dose
(P<.001). For patients for whom labetalol was ordered, patients
receiving at least 1 dose had an LOS of 11.8 days vs 7.9 days for
those who did not receive a dose (P<.001).\footnote{Weder \emph{et al.} J
  Clin Hypertens (Greenwich). 2010 Jan;12(1):29--33. }

No substantial benefit from emergency department referral compared
with sending the patient home from the office for outpatient
management of blood pressure.\footnote{Patel \emph{et al.} JAMA Intern Med.
  2016;176(7):981. PMID 27294333.}

Acute tx vs not: no differences in return to the emergency department
at 24 hours and at 30 days, and there were no differences in mortality
at 30 days and at one year.\footnote{Levy \emph{et al.} Am J Emerg Med. 2015
  Sep;33(9):1219--24. PMID 26087706.}

An RCT. There was no difference between groups 1 and 2 in the time to
BP control, nor was there a difference at 24 hours in pressure
reduction between groups 1,2, or 3.\footnote{Zeller \emph{et al.} Arch Intern
  Med. 1989;149(10):2186. PMID 2679473.}



\end{document}

% LocalWords:  nonemergent mmHg HTN Curr Cardiol Weder Whelton ACC reelevation
% LocalWords:  Katz Clin Hypertens JAMA Emerg Sep RCT Zeller alls reated
