\documentclass{tufte-handout}
\title{Inpatient Hypertension Management}
\author{Andrew Zimolzak, MD, MMSc}
\date{April 7, 2018}

\begin{document}

\maketitle


\section{Hypertensive Urgency}

\marginnote{Licensed under Creative Commons BY-NC-SA 4.0.
You are free to share and adapt this material for noncommercial
purposes if you credit the author and distribute under the same license.}

Where does the threshold start, where did it used to start?
What meds would I use? I
don't advocate IV. What would I check for or think about regarding
\emph{emergency?} Steep rise should make you more likely to think of
emergency. Compare to somebody who is ``used to'' that high pressure.

Choice quote: ``Many of these
patients have withdrawn from or are noncompliant with antihypertensive therapy\ldots{}.
[T]reated by reinstitution or intensification of antihypertensive drug
therapy and treatment of anxiety\ldots{}. There is no indication for referral to the emergency
department, immediate reduction in BP in the emergency department, or hospitalization for such
patients.''\footnote{Whelton et al. 2017 ACC (and 10 other societies) Guideline for the Prevention, Detection, Evaluation, and Management of High Blood Pressure in Adults. Hypertension. 2017 Nov 13. This is 192 pages with > 200 pp.\ of supplements. Pp. 137--143 are about emergency/urgency.}


\section{Otherwise}

Holding and restarting meds on admission/discharge.
What BP numbers to target. What do the guidelines
say? What risk are we reducing (what time baseline)?
Heart failure (if bad), even compensated, is different.
How long do various agents take? ACE ARB mnemonics. Beta. Calcium.
Others like clonidine. Which classes are good for ``treating a number?''

\end{document}
